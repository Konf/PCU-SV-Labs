\chapter{Контроллер \eng{PS/2} для клавиатуры}

\par{Для успешного применения цифровых устройств в реальности важно обеспечить их взаимодействие с внешним миром. Для этого существуют внешние интерфейсы для обмена информацией с другими устройствами и приборами. Внешние интерфейсы имеют различное функциональное назначение и спецификацию. Например, раньше для подключения клавиатуры и мыши использовался интерфейс \eng{PS/2}. Один с самых простых способов подключения устройств ввода-вывода.}

\vspace{4mm}

\par{Давайте изучим протокол и реализацию интерфейса \eng{PS/2}.}

\par{Интерфейс \eng{PS/2} сточки зрения соединения представляет собой два провода \kword{Clock} и \kword{Data}. По \kword{Clock} передаются синхроимпульсы, а по \kword{Data} предаются данные. На рисунке пример одной транзакции обмена.}

\begin{figure}[H]
	\centering
	\def\svgwidth{10cm}
	\includesvg{images/lab_6/diag}
	\caption{Передача данных по протоколу PS/2}
\end{figure}

\par{Структура транзакции похожа на \eng{UART} и состоит из:}
  \begin{enumerate}%[noitemsep,topsep=0pt, after=\vspace{2pt}]
    \item старт бит – всегда ноль;
    \item 8 бит данных;
    \item бит четности, равен 1 если количество единиц в данных четно и 0 если нечетно;
    \item стоп бит – всегда единица.
  \end{enumerate}

\par{Данные на линию выставляются, когда \kword{Clock} равен \textbf{1} и считываются, когда \kword{Clock} равен \textbf{0}. На практике данные обычно выставляются по положительному фронту и считываются по отрицательному.}

\par{Частота сигнала \kword{Clock} примерно 10-16.7кГц. Время от фронта сигнала \kword{Clock} до момента изменения сигнала \kword{Data} не менее 5 микросекунд.}

\vspace{4mm}

\par{Транзакции разделяются на два вида:}
  \begin{enumerate}%[noitemsep,topsep=0pt, after=\vspace{2pt}]
    \item От устройства к контроллеру;
    \item От контроллера к устройству.
  \end{enumerate}

\par{На примере клавиатуры.}
  \begin{enumerate}%[noitemsep,topsep=0pt, after=\vspace{2pt}]
    \item Клавиатура посылает на контроллер восьмибитный код нажатой клавиши;
    \item Контроллер посылает на клавиатуру команды управления. Такие как, команды сброса, выключения светодиодов.
  \end{enumerate}

\begin{figure}[H]
	\centering
	\def\svgwidth{10cm}
	\includesvg{images/lab_6/diag}
	\caption{Передача восьмибитного пакета данных}
\end{figure}

\par{При транзакции от устройства (клавиатуры) к контроллеру сигналы на линиях \kword{Clock} и \kword{Data} генерирует устройство. Контроллер выступает в роли приемника считывая данные по отрицательному фронту \kword{Clock}.}

\par{При передаче в обратную сторону команд от контроллера к клавиатуре или мыши протокол отличается от описанного выше.}

\par{Последовательность обмена другая:}
  \begin{enumerate}%[noitemsep,topsep=0pt, after=\vspace{2pt}]
    \item контроллер опускает сигнал \kword{Clock} в ноль на время примерно 100 микросекунд;
    \item контроллер опускает сигнал \kword{Data} в ноль формируя старт бит;
    \item контроллер отпускает сигнал \kword{Clock} в логическую единицу, клавиатура фиксирует старт бит;
    \item далее клавиатура генерирует сигнал \kword{Clock}, а хост контроллер подает передаваемые биты;
    \item после того, как контроллер передал все свои биты, включая бит четности и стоп бит, клавиатура посылает последний бит «ноль», который является подтверждением приема.
  \end{enumerate}

\begin{figure}[H]
	\centering
	\def\svgwidth{10cm}
	\includesvg{images/lab_6/diag_r}
	\caption{Распределение управления между контроллером и устройством}
\end{figure}

\par{Поскольку одним сигналом управляют два устройства, то довольно трудно понять, кто в какой момент времени управляет сигналом. По этому, диаграмма нарисована двумя цветами. Красный цвет – сигнал управляется контроллером, а синий – сигнал управляется устройством.}

\vspace{4mm}

\par{Давайте выясним какие коды же коды передает клавиатура для каждой клавиши.}

\par{Ниже приведена таблица кодов для клавиш. Каждой клавише соответствует код генерируемый при нажатии и код генерируемый при деактивации. Коды состоящие из нескольких байтов предаются в нескольких подряд идущих транзакций.}

\vspace{4mm}

% \begin{center}
%   \begin{tabular}{ | l | l | l | l | l | l | }
%     \hline
% 	    \kword{KEY} & \kword{MAKE} & \kword{BREAK} & \kword{KEY} & \kword{MAKE} & \kword{BREAK} \\ \hline
% 	    A & 1C & F0,1C & R ALT & E0,11 & E0,F0,11  \\ \hline
% 	    B & 32 & F0,32 & APPS  & E0,2F & E0,F0,2F  \\ \hline
% 	    C & 21 & F0,21 & ENTER & 5A & F0,5A        \\ \hline
% 	    D & 23 & F0,23 & ESC   & 76 & F0,76        \\ \hline
% 	    E & 24 & F0,24 & F1 & 05 & F0,05           \\ \hline
% 	    F & 2B & F0,2B & F2 & 06 & F0,06           \\ \hline
%   \end{tabular}
% \end{center}

\begin{center}
  \begin{tabular}{ | l | l | l | l | l | l | }
    \hline
	    \kword{KEY} & \kword{MAKE} & \kword{BREAK} & \kword{KEY} & \kword{MAKE} & \kword{BREAK} \\ \hline
	    A & 1C & F0,1C & R ALT & E0,11 & E0,F0,11  \\ \hline
	    B & 32 & F0,32 & APPS  & E0,2F & E0,F0,2F  \\ \hline
	    C & 21 & F0,21 & ENTER & 5A & F0,5A        \\ \hline
	    D & 23 & F0,23 & ESC   & 76 & F0,76        \\ \hline
	    E & 24 & F0,24 & F1 & 05 & F0,05           \\ \hline
	    F & 2B & F0,2B & F2 & 06 & F0,06           \\ \hline
	    G & 34 & F0,34 & F3 & 04 & F0,04           \\ \hline
	    H & 33 & F0,33 & F4 & 0C & F0,0C           \\ \hline	    
	    I & 43 & F0,43 & F5 & 03 & F0,03           \\ \hline
	    J & 3B & F0,3B & F6 & 0B & F0,0B           \\ \hline
	    K & 42 & F0,42 & F7 & 83 & F0,83           \\ \hline
	    L & 4B & F0,4B & F8 & 0A & F0,0A           \\ \hline
	    M & 3A & F0,3A & F9 & 01 & F0,01           \\ \hline
	    N & 31 & F0,31 & F10 & 09 & F0,09          \\ \hline
	    O & 44 & F0,44 & F11 & 78 & F0,78          \\ \hline
	    P & 4D & F0,4D & F12 & 07 & F0,07          \\ \hline
	    Q & 15 & F0,15 & SCROLL & 7E & F0,7E       \\ \hline
	    R & 2D & F0,2D & [ & 54 & FO,54            \\ \hline
	    S & 1B & F0,1B & INSERT & E0,70 & E0,F0,70 \\ \hline
	    T & 2C & F0,2C & HOME & E0,6C & E0,F0,6C   \\ \hline
	    U & 3C & F0,3C & PG UP & E0,7D & E0,F0,7D  \\ \hline
	    V & 2A & F0,2A & DELETE & E0,71 & E0,F0,71 \\ \hline
	    W & 1D & F0,1D & END & E0,69 & E0,F0,69    \\ \hline
	    X & 22 & F0,22 & PG DN & E0,7A & E0,F0,7A  \\ \hline
	    Y & 35 & F0,35 & UP & E0,75 & E0,F0,75     \\ \hline
	    Z & 1A & F0,1A & LEFT & E0,6B & E0,F0,6B   \\ \hline
	    0 & 45 & F0,45 & DOWN &  E0,72 & E0,F0,72  \\ \hline
	    1 & 16 & F0,16 & RIGHT &  E0,74 & E0,F0,74 \\ \hline
	    2 & 1E & F0,1E & NUM & 77 & F0,77          \\ \hline
	    3 & 26 & F0,26 & KP / & E0,4A & E0,F0,4A   \\ \hline
	    4 & 25 & F0,25 & KP * & 7C & F0,7C         \\ \hline
	    5 & 2E & F0,2E & KP - & 7B & F0,7B         \\ \hline
	    6 & 36 & F0,36 & KP + & 79 & F0,79         \\ \hline
	    7 & 3D & F0,3D & KP EN & E0,5A & E0,F0,5A  \\ \hline
	    8 & 3E & F0,3E & KP . & 71 & F0,71         \\ \hline
	    9 & 46 & F0,46 & KP 0 & 70 & F0,70         \\ \hline
	    ` & 0E & F0,0E & KP 1 & 69 & F0,69         \\ \hline
	    - & 4E & F0,4E & KP 2 & 72 & F0,72         \\ \hline
	    = & 55 & F0,55 & KP 3 & 7A & F0,7A         \\ \hline
	    \ & 5D & F0,5D & KP 4 & 6B & F0,6B         \\ \hline
	    BKSP &   66 & F0,66 & KP 5 & 73 & F0,73    \\ \hline
	    SPACE &  29 & F0,29 & KP 6 & 74 & F0,74    \\ \hline
  \end{tabular}
\end{center}

\begin{center}
  \begin{tabular}{ | l | l | l | l | l | l | }
    \hline
	    \kword{KEY} & \kword{MAKE} & \kword{BREAK} & \kword{KEY} & \kword{MAKE} & \kword{BREAK} \\ \hline
	    TAB  &   0D & F0,0D & KP 7 & 6C & F0,6C    \\ \hline
	    CAPS  &  58 & F0,58 & KP 8 & 75 & F0,75    \\ \hline
	    L SHFT & 12 & FO,12 & KP 9 & 7D & F0,7D    \\ \hline
	    L CTRL & 14 & FO,14 & ] & 5B & F0,5B       \\ \hline
	    L GUI  & E0,1F & E0,F0,1F & ; & 4C & F0,4C \\ \hline
	    L ALT  & 11 & F0,11 & ' & 52 & F0,52       \\ \hline
	    R SHFT & 59 & F0,59 & , & 41 & F0,41       \\ \hline
	    R CTRL & E0,14 & E0,F0,14 & . & 49 & F0,49 \\ \hline
	    R GUI  & E0,27 & E0,F0,27 & / & 4A & F0,4A \\ \hline
  \end{tabular}
\end{center}

\vspace{4mm}

\par{Разработаем простой контроллер для клавиатуры. Контроллер предназначается для работы только в режиме устройство-контроллер.}

\par{Для начала определим функционал контроллера и набор сигналов.}

\par{Функционал:}
  \begin{enumerate}%[noitemsep,topsep=0pt, after=\vspace{2pt}]
    \item Считываются данные с линии \eng{PS/2} по отрицательному фронту синхросигнала;
    \item Проверяется корректность принятых данных. Проверяется стоп бит, бит четности и стартовый бит;
    \item Выводит принятые данные на внешнюю шину и формирует сигнал готовности данных.
  \end{enumerate}

\par{Набор сигналов:}
  \begin{enumerate}%[noitemsep,topsep=0pt, after=\vspace{2pt}]
    \item \kword{areset} – асинхронный сброс, активный уровень \textbf{1};
    \item \kword{clk\_50} – вход тактовой частоты с частотой 50 МГц;
    \item \kword{\kword{Data}} – восьмибитная шина данных;
    \item \kword{valid\_\kword{Data}} – сигнал готовности данных, равен \textbf{1} с момента завершения приема по \eng{PS/2} до начала следующей транзакции;
    \item \kword{ps2\_clk} – тактовая линия \eng{PS/2}, является внешним сигналом для ПЛИС;
    \item \kword{ps2\_dat} – сигнальная линия \eng{PS/2}, является внешним сигналом для ПЛИС.
  \end{enumerate}

\par{\textbf{Примечание}. Сигналы \kword{ps2\_clk} и \kword{ps2\_dat} как внешние сигналы необходимо подключить к соответствующим пинам ПЛИС. Это пины \eng{H15(\kword{ps2\_clk}) $ $ J14(\kword{ps2\_dat})}, к которым на плате \eng{DE1} подключен коннектор \eng{PS/2}.}

%\lstinputlisting[caption={Описание входов и выходов контроллера}]{./code_examples/lab_6/in_out.v}

\begin{listing}[H]
	\inputminted{SystemVerilog}{code_examples/lab_6/in_out.sv}
	\caption{Описание входов и выходов контроллера}
	%\label{lab2:code1}
\end{listing}


\par{Прежде всего, необходимо надежно определять нисходящий фронт сигнала \kword{ps2\_clk}, так как его качество (крутизна фронтов, зашумленность) может сильно варьироваться в зависимости от клавиатуры и непосредственное тактирование от этого сигнала может вызвать некорректную работу всей схемы в целом.}

\par{Для определения нисходящего фронта мы используем вариант схемы для работы с кнопками. Схема представляет из себя десятибитный сдвиговый регистр, в который каждый такт \kword{clk\_50} сдвигается текущее значение \kword{ps2\_clk}. Схема ожидает момент, когда старшие 5 бит сдвигового регистра заполнены нулями, а младшие 5 бит -- единицами. В этот момент на 1 такт возводится сигнал \kword{ps2\_clk\_negedge}, который используется в остальной схеме.}

%\lstinputlisting[caption={Описание схемы определения нисходящего фронта сигнала ps2\_clk}]{./code_examples/lab_6/clk_detect.v}


\begin{listing}[H]
	\inputminted{SystemVerilog}{code_examples/lab_6/clk_detect.sv}
	\caption{Описание схемы определения нисходящего фронта сигнала ps2\_clk}
	%\label{lab2:code1}
\end{listing}


\par{Основой контроллера будет конечный автомат со следующей последовательностью действий:}
  \begin{enumerate}%[noitemsep,topsep=0pt, after=\vspace{2pt}]
    \item Состояние покоя;
    \item Прием стартового бита и его проверка. Если стартовый бит не равен 0 переходим в состояние покоя;
    \item Прием данных;
    \item Прием бита четности и стопового бита их проверка.
    \item При правильных значениях стопового бита и бита четности формирования сигнала готовности данных.
    \item Переход в состояние покоя.
  \end{enumerate}

%\lstinputlisting[caption={Описание конечного автомата контроллера}]{./code_examples/lab_6/state_m.v}


\begin{listing}[H]
	\inputminted{SystemVerilog}{code_examples/lab_6/state_m.sv}
	\caption{Описание конечного автомата контроллера}
	%\label{lab2:code1}
\end{listing}


\par{Конечный автомат имеет три состояния \eng{IDLE, RECEIVE\_DATA, CHECK\_PARITY\_STOP\_BIT}.}

\par{\eng{IDLE} – состояние покоя в котором контроллер ожидает первого отрицательного фронта \kword{ps2\_clk}. Переход в состояние \eng{RECEIVE\_DATA} происходит по отрицательному фронту \kword{ps2\_clk} если \kword{ps2\_dat} равно 0, то есть пришел стартовый бит, иначе остаемся в \eng{IDLE}.}

\par{\eng{RECEIVE\_DATA} – состояние в ходе, которого происходит прием данных и бита четности. Переход в состояние \eng{CHECK\_PARITY\_STOP\_BIT} происходит при счетчике бит равном 8. Отсчитано 8 бит данных и идет прием бита четности.}

\par{Последнее состояние \eng{CHECK\_PARITY\_STOP\_BITS} длительностью в один период \kword{ps2\_clk}. В \eng{CHECK\_PARITY\_STOP\_BITS} происходит проверка бита паритета и стопового бита.}

%\lstinputlisting[caption={Описание сдвигового регистра}]{./code_examples/lab_6/shift.v}


\begin{listing}[H]
	\inputminted{SystemVerilog}{code_examples/lab_6/shift.sv}
	\caption{Описание сдвигового регистра}
	%\label{lab2:code1}
\end{listing}

\par{Сдвиговый регистр необходим для приема и хранения данных и бита четности. По этому, разрядность регистра равна 9. По завершению транзакции в восьмом бите хранится бит четности с 7-0 бит данные. Данные непрерывным присваиванием выведены на внешнюю шину модуля.}

\par{Если обратиться к началу лабораторной работы то стоит заметить что данные передаются по интерфейсу \eng{PS/2} начиная с младшего бита. Логично будет использовать схему работы сдвигового регистра, при которой сдвиг происходит вправо. Таким образом, первый принятый бит окажется в 0 ячейке сдвигового регистра по окончании транзакции.}

\par{Запись в сдвиговый регистр происходит по отрицательному фронту \kword{ps2\_clk} и состоянию конечного автомата \eng{RECEIVE\_DATA}.}

%\lstinputlisting[caption={Описание счетчика принятых бит}]{./code_examples/lab_6/count_bit.v}

\begin{listing}[H]
	\inputminted{SystemVerilog}{code_examples/lab_6/count_bit.sv}
	\caption{Описание счетчика принятых бит}
	%\label{lab2:code1}
\end{listing}


\par{Счетчик принятых бит служит для контроля за процессом приема. Инкрементация счетчика происходит только в состоянии \eng{RECEIVE\_DATA}.}

%\lstinputlisting[caption={Описание вырабатывания сигнала готовности к передаче}]{./code_examples/lab_6/parity.v}

\begin{listing}[H]
	\inputminted{SystemVerilog}{code_examples/lab_6/parity.sv}
	\caption{Описание вырабатывания сигнала готовности к передаче}
	%\label{lab2:code1}
\end{listing}

\par{Последним нерассмотренным моментом остался вопрос генерации сигнала готовности данных. Как было сказано ранее сигнал готовности генерируется к конце транзакции в случае успешного приема и равен 1 до начала следующей транзакции. То есть пока конечный автомат в состоянии \eng{IDLE}.}

\par{Условием успешного окончания транзакции является стоповый бит равный 1 и бит четности равный рассчитанному значению.}

\par{Генерация сигнала готовности происходит в момент приема стопового бита. По этому для его проверки достаточно убедиться что значение на линии \kword{ps2\_dat} равно 1.}

\par{Для проверки бита четности необходимо рассчитать четность принятых 8 бит данных и сравнить ее с значением бита четности. Для упрощения читаемости кода используется функция расчета честности для 8 разрядного регистра согласно правилу отрицания побитового \eng{XOR} регистра. Правило расчета бита паритета можно узнать из стандарта на интерфейс \eng{PS/2}.}

\section{Полное описание контроллера PS/2}

%\lstinputlisting[caption={Пример реализации контроллера PS/2}]{./code_examples/lab_6/ps2.v}



\begin{longlisting}
	\inputminted{SystemVerilog}{code_examples/lab_6/ps2.sv}
	\caption{Пример реализации контроллера PS/2}
	%\label{lab2:code1}
\end{longlisting}

\section{Задание лабораторной работы:}
\begin{enumerate}%[noitemsep,topsep=0pt, after=\vspace{2pt}]

\item{Ознакомиться с спецификацией на интерфейс \eng{PS/2} и представленной реализацией контроллера клавиатуры.}

%\item{Построить временные диаграммы его работы с помощью САПР \eng{Altera Quartus}.}

\item{Подготовиться к выполнению индивидуального задания с использованием предложенного контроллера на лабораторной работе.}

\end{enumerate}





\section{Варианты индивидуальных заданий}


\begin{enumerate}
  
  \setlength\itemsep{1em}

  \item{
  Выводить на семисегментные индикаторы только коды клавиш ''W'', ''A'', ''S'', ''D'' и ''пробел''.
  }

  \item{
  Выводить на семисегментные индикаторы только коды клавиш ''1'', ''3'', ''5'', ''7'', ''9''.
  }

  \item{
  Выводить на семисегментные индикаторы только коды клавиш, находящихся на num-pad.
  }

  \item{
  Выводить на семисегментные индикаторы только коды клавиш ''Q'', ''W'', ''E'', ''R'', ''T'', ''Y''.
  }

  \item{
  Выводить на семисегментные индикаторы только коды клавиш ''I'', ''D'', ''Q'', ''D''.
  }

  \item{
  Выводить на семисегментные индикаторы только коды клавиш со стрелками.
  }

  \item{
  Выводить на семисегментные индикаторы только коды клавиш ''I'', ''D'', ''K'', ''F'', ''A''.
  }

  \item{
  Выводить на семисегментные индикаторы только коды клавиш F1 -- F12.
  }

  \item{
  Выводить на семисегментные индикаторы только коды клавиш ''Z'', ''X'', ''C'', ''V'', ''B'', ''N''.
  }

  \item{
  Выводить на семисегментные индикаторы только коды клавиш ''L'', ''J'', ''S'', ''P'', ''Q'', ''K''.
  }

\end{enumerate}
